\chapter{需求分析}
\label{chap:intro}

这是山西农业大学软件学院非官方的毕业设计文档 \LaTeX 模板,当前版本是 \version 。

\section{使用模板}

\subsection{准备工作}
\label{sec:requirements}

要使用这个模板撰写学位论文,需要在\emph{TeX系统}、\emph{中英文字体}、\emph{TeX技能}上有所准备。

\begin{itemize}[noitemsep,topsep=0pt,parsep=0pt,partopsep=0pt]
	\item 系统:所使用的系统要支持 引擎,且带有ctex 2.x宏包,以2015年的\emph{完整}TeXLive、MacTeX发行版为佳。
	\item 中英文字体:操作系统中需要安装TeX Gyre Termes字体和四款Adobe中文字体:AdobeSongStd、AdobeKaitiStd、AdobeHeitiStd、AdobeFangsongStd。
	\item TeX技能:尽管提供了对模板的必要说明,但这不是一份“ \LaTeX 入门文档”。在使用前请先通读其他入门文档。
	\item 针对Windows用户的额外需求:学位论文模本分别使用git和GNUMake进行版本控制和构建,建议从Cygwin安装这两个工具。
\end{itemize}

\section{使用模板}

\subsection{准备工作}
\label{sec:requirements}

要使用这个模板撰写学位论文,需要在\emph{TeX系统}、\emph{中英文字体}、\emph{TeX技能}上有所准备。

\begin{itemize}[noitemsep,topsep=0pt,parsep=0pt,partopsep=0pt]
	\item 系统:所使用的系统要支持 引擎,且带有ctex 2.x宏包,以2015年的\emph{完整}TeXLive、MacTeX发行版为佳。
	\item 中英文字体:操作系统中需要安装TeX Gyre Termes字体和四款Adobe中文字体:AdobeSongStd、AdobeKaitiStd、AdobeHeitiStd、AdobeFangsongStd。
	\item TeX技能:尽管提供了对模板的必要说明,但这不是一份“ \LaTeX 入门文档”。在使用前请先通读其他入门文档。
	\item 针对Windows用户的额外需求:学位论文模本分别使用git和GNUMake进行版本控制和构建,建议从Cygwin安装这两个工具。
\end{itemize}

\section{使用模板}

\subsection{准备工作}
\label{sec:requirements}

要使用这个模板撰写学位论文,需要在\emph{TeX系统}、\emph{中英文字体}、\emph{TeX技能}上有所准备。

\begin{itemize}[noitemsep,topsep=0pt,parsep=0pt,partopsep=0pt]
	\item 系统:所使用的系统要支持 引擎,且带有ctex 2.x宏包,以2015年的\emph{完整}TeXLive、MacTeX发行版为佳。
	\item 中英文字体:操作系统中需要安装TeX Gyre Termes字体和四款Adobe中文字体:AdobeSongStd、AdobeKaitiStd、AdobeHeitiStd、AdobeFangsongStd。
	\item TeX技能:尽管提供了对模板的必要说明,但这不是一份“ \LaTeX 入门文档”。在使用前请先通读其他入门文档。
	\item 针对Windows用户的额外需求:学位论文模本分别使用git和GNUMake进行版本控制和构建,建议从Cygwin安装这两个工具。
\end{itemize}

\subsection{准备工作}
\label{sec:requirements}

要使用这个模板撰写学位论文,需要在\emph{TeX系统}、\emph{中英文字体}、\emph{TeX技能}上有所准备。

\begin{itemize}[noitemsep,topsep=0pt,parsep=0pt,partopsep=0pt]
	\item 系统:所使用的系统要支持 引擎,且带有ctex 2.x宏包,以2015年的\emph{完整}TeXLive、MacTeX发行版为佳。
	\item 中英文字体:操作系统中需要安装TeX Gyre Termes字体和四款Adobe中文字体:AdobeSongStd、AdobeKaitiStd、AdobeHeitiStd、AdobeFangsongStd。
	\item TeX技能:尽管提供了对模板的必要说明,但这不是一份“ \LaTeX 入门文档”。在使用前请先通读其他入门文档。
	\item 针对Windows用户的额外需求:学位论文模本分别使用git和GNUMake进行版本控制和构建,建议从Cygwin安装这两个工具。
\end{itemize}

\subsection{准备工作}
\label{sec:requirements}

要使用这个模板撰写学位论文,需要在\emph{TeX系统}、\emph{中英文字体}、\emph{TeX技能}上有所准备。

\begin{itemize}[noitemsep,topsep=0pt,parsep=0pt,partopsep=0pt]
	\item 系统:所使用的系统要支持 引擎,且带有ctex 2.x宏包,以2015年的\emph{完整}TeXLive、MacTeX发行版为佳。
	\item 中英文字体:操作系统中需要安装TeX Gyre Termes字体和四款Adobe中文字体:AdobeSongStd、AdobeKaitiStd、AdobeHeitiStd、AdobeFangsongStd。
	\item TeX技能:尽管提供了对模板的必要说明,但这不是一份“ \LaTeX 入门文档”。在使用前请先通读其他入门文档。
	\item 针对Windows用户的额外需求:学位论文模本分别使用git和GNUMake进行版本控制和构建,建议从Cygwin安装这两个工具。
\end{itemize}

\subsection{准备工作}
\label{sec:requirements}

要使用这个模板撰写学位论文,需要在\emph{TeX系统}、\emph{中英文字体}、\emph{TeX技能}上有所准备。

\begin{itemize}[noitemsep,topsep=0pt,parsep=0pt,partopsep=0pt]
	\item 系统:所使用的系统要支持 引擎,且带有ctex 2.x宏包,以2015年的\emph{完整}TeXLive、MacTeX发行版为佳。
	\item 中英文字体:操作系统中需要安装TeX Gyre Termes字体和四款Adobe中文字体:AdobeSongStd、AdobeKaitiStd、AdobeHeitiStd、AdobeFangsongStd。
	\item TeX技能:尽管提供了对模板的必要说明,但这不是一份“ \LaTeX 入门文档”。在使用前请先通读其他入门文档。
	\item 针对Windows用户的额外需求:学位论文模本分别使用git和GNUMake进行版本控制和构建,建议从Cygwin安装这两个工具。
\end{itemize}

\subsection{准备工作}
\label{sec:requirements}

要使用这个模板撰写学位论文,需要在\emph{TeX系统}、\emph{中英文字体}、\emph{TeX技能}上有所准备。

\begin{itemize}[noitemsep,topsep=0pt,parsep=0pt,partopsep=0pt]
	\item 系统:所使用的系统要支持 引擎,且带有ctex 2.x宏包,以2015年的\emph{完整}TeXLive、MacTeX发行版为佳。
	\item 中英文字体:操作系统中需要安装TeX Gyre Termes字体和四款Adobe中文字体:AdobeSongStd、AdobeKaitiStd、AdobeHeitiStd、AdobeFangsongStd。
	\item TeX技能:尽管提供了对模板的必要说明,但这不是一份“ \LaTeX 入门文档”。在使用前请先通读其他入门文档。
	\item 针对Windows用户的额外需求:学位论文模本分别使用git和GNUMake进行版本控制和构建,建议从Cygwin安装这两个工具。
\end{itemize}

\subsection{准备工作}
\label{sec:requirements}

要使用这个模板撰写学位论文,需要在\emph{TeX系统}、\emph{中英文字体}、\emph{TeX技能}上有所准备。

\begin{itemize}[noitemsep,topsep=0pt,parsep=0pt,partopsep=0pt]
	\item 系统:所使用的系统要支持 引擎,且带有ctex 2.x宏包,以2015年的\emph{完整}TeXLive、MacTeX发行版为佳。
	\item 中英文字体:操作系统中需要安装TeX Gyre Termes字体和四款Adobe中文字体:AdobeSongStd、AdobeKaitiStd、AdobeHeitiStd、AdobeFangsongStd。
	\item TeX技能:尽管提供了对模板的必要说明,但这不是一份“ \LaTeX 入门文档”。在使用前请先通读其他入门文档。
	\item 针对Windows用户的额外需求:学位论文模本分别使用git和GNUMake进行版本控制和构建,建议从Cygwin安装这两个工具。
\end{itemize}

\begin{lstlisting}[language={C}, caption={一段C源代码}]
#include <stdio.h>
#include <unistd.h>
#include <sys/types.h>
#include <sys/wait.h>

int main() {
  pid_t pid;

  switch ((pid = fork())) {
  case -1:
    printf("fork failed\n");
    break;
  case 0:
    /* child calls exec */
    execl("/bin/ls", "ls", "-l", (char*)0);
    printf("execl failed\n");
    break;
  default:
    /* parent uses wait to suspend execution until child finishes */
    wait((int*)0);
    printf("is completed\n");
    break;
  }

  return 0;
}
\end{lstlisting}

\subsection{准备工作}
\label{sec:requirements}

要使用这个模板撰写学位论文,需要在\emph{TeX系统}、\emph{中英文字体}、\emph{TeX技能}上有所准备。

\begin{itemize}[noitemsep,topsep=0pt,parsep=0pt,partopsep=0pt]
	\item 系统:所使用的系统要支持 引擎,且带有ctex 2.x宏包,以2015年的\emph{完整}TeXLive、MacTeX发行版为佳。
	\item 中英文字体:操作系统中需要安装TeX Gyre Termes字体和四款Adobe中文字体:AdobeSongStd、AdobeKaitiStd、AdobeHeitiStd、AdobeFangsongStd。
	\item TeX技能:尽管提供了对模板的必要说明,但这不是一份“ \LaTeX 入门文档”。在使用前请先通读其他入门文档。
	\item 针对Windows用户的额外需求:学位论文模本分别使用git和GNUMake进行版本控制和构建,建议从Cygwin安装这两个工具。
\end{itemize}

\begin{figure}[!htp]
  \centering
  \includegraphics[width=0.3\textwidth]{sxau.jpg}
  \hspace{1cm}
  \includegraphics[width=0.3\textwidth]{sxau.jpg}
  \caption{xx\label{xx}}
%  \bicaption[fig:SRR]{这里将出现在插图索引中}{中文题图}{Fig}{English caption}
\end{figure}

\begin{figure}[!htp]
  \centering
  \includegraphics[width=0.3\textwidth]{sxau.jpg}
  \caption{xx\label{xx}}
%  \bicaption[fig:SRR]{这里将出现在插图索引中}{中文题图}{Fig}{English caption}
\end{figure}

\begin{table}[!hpb]
  \centering
  \caption{xx\label{xx}}
  
  % \bicaption[tab:firstone]{指向一个表格的表目录索引}{一个颇为标准的三线表格}{Table}{A Table}
  \begin{tabular}{@{}llr@{}} \toprule
    类别 & 水溶型 & 胶体分散型 (\$)\\ \midrule
    状态 & 溶解~胶束 & 分散 \\
    & 透明 & 半透明 \\
    重均分子量 & 1000-10000 & 数千-20万 \\ \bottomrule
  \end{tabular}
\end{table}

\subsection{准备工作}
\label{sec:requirements}

要使用这个模板撰写学位论文\parencite{Meta_CN,JohnD,IEEE-1363},需要在\emph{TeX系统}、\emph{中英文字体}、\emph{TeX技能}上有所准备。

\begin{itemize}[noitemsep,topsep=0pt,parsep=0pt,partopsep=0pt]
	\item 系统:所使用的系统要支持 引擎,且带有ctex 2.x宏包,以2015年的\emph{完整}TeXLive、MacTeX发行版为佳。
	\item 中英文字体:操作系统中需要安装TeX Gyre Termes字体和四款Adobe中文字体:AdobeSongStd、AdobeKaitiStd、AdobeHeitiStd、AdobeFangsongStd。
	\item TeX技能:尽管提供了对模板的必要说明,但这不是一份“ \LaTeX 入门文档”。在使用前请先通读其他入门文档。
	\item 针对Windows用户的额外需求:学位论文模本分别使用git和GNUMake进行版本控制和构建,建议从Cygwin安装这两个工具。
\end{itemize}

\begin{references}{99}
    \item\textsc{Chen H, Chan C~T}. {Acoustic cloaking in three dimensions using acoustic metamaterials}[J]. Applied Physics Letters, 2007, 91:183518.
    \item\textsc{Chen H, Wu B~I, Zhang B}, et al. {Electromagnetic Wave Interactions with a Metamaterial Cloak}[J]. Physical Review Letters, 2007, 99(6):63903.
\end{references}


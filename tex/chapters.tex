\chapter{需求分析}
\label{chap:intro}

这是山西农业大学软件学院非官方的毕业设计文档 \LaTeX 模板,当前版本是 \version 。

\section{使用模板}

\subsection{准备工作}
\label{sec:requirements}

要使用这个模板撰写学位论文,需要在\emph{TeX系统}、\emph{中英文字体}、\emph{TeX技能}上有所准备。

\begin{itemize}[noitemsep,topsep=0pt,parsep=0pt,partopsep=0pt]
	\item 系统:所使用的系统要支持 引擎,且带有ctex 2.x宏包,以2015年的\emph{完整}TeXLive、MacTeX发行版为佳。
	\item 中英文字体:操作系统中需要安装TeX Gyre Termes字体和四款Adobe中文字体:AdobeSongStd、AdobeKaitiStd、AdobeHeitiStd、AdobeFangsongStd。
	\item TeX技能:尽管提供了对模板的必要说明,但这不是一份“ \LaTeX 入门文档”。在使用前请先通读其他入门文档。
	\item 针对Windows用户的额外需求:学位论文模本分别使用git和GNUMake进行版本控制和构建,建议从Cygwin安装这两个工具。
\end{itemize}

\section{使用模板}

\subsection{准备工作}
\label{sec:requirements}

要使用这个模板撰写学位论文,需要在\emph{TeX系统}、\emph{中英文字体}、\emph{TeX技能}上有所准备。

\begin{itemize}[noitemsep,topsep=0pt,parsep=0pt,partopsep=0pt]
	\item 系统:所使用的系统要支持 引擎,且带有ctex 2.x宏包,以2015年的\emph{完整}TeXLive、MacTeX发行版为佳。
	\item 中英文字体:操作系统中需要安装TeX Gyre Termes字体和四款Adobe中文字体:AdobeSongStd、AdobeKaitiStd、AdobeHeitiStd、AdobeFangsongStd。
	\item TeX技能:尽管提供了对模板的必要说明,但这不是一份“ \LaTeX 入门文档”。在使用前请先通读其他入门文档。
	\item 针对Windows用户的额外需求:学位论文模本分别使用git和GNUMake进行版本控制和构建,建议从Cygwin安装这两个工具。
\end{itemize}

\section{使用模板}

\subsection{准备工作}
\label{sec:requirements}

要使用这个模板撰写学位论文,需要在\emph{TeX系统}、\emph{中英文字体}、\emph{TeX技能}上有所准备。

\begin{itemize}[noitemsep,topsep=0pt,parsep=0pt,partopsep=0pt]
	\item 系统:所使用的系统要支持 引擎,且带有ctex 2.x宏包,以2015年的\emph{完整}TeXLive、MacTeX发行版为佳。
	\item 中英文字体:操作系统中需要安装TeX Gyre Termes字体和四款Adobe中文字体:AdobeSongStd、AdobeKaitiStd、AdobeHeitiStd、AdobeFangsongStd。
	\item TeX技能:尽管提供了对模板的必要说明,但这不是一份“ \LaTeX 入门文档”。在使用前请先通读其他入门文档。
	\item 针对Windows用户的额外需求:学位论文模本分别使用git和GNUMake进行版本控制和构建,建议从Cygwin安装这两个工具。
\end{itemize}

\subsection{准备工作}
\label{sec:requirements}

要使用这个模板撰写学位论文,需要在\emph{TeX系统}、\emph{中英文字体}、\emph{TeX技能}上有所准备。

\begin{itemize}[noitemsep,topsep=0pt,parsep=0pt,partopsep=0pt]
	\item 系统:所使用的系统要支持 引擎,且带有ctex 2.x宏包,以2015年的\emph{完整}TeXLive、MacTeX发行版为佳。
	\item 中英文字体:操作系统中需要安装TeX Gyre Termes字体和四款Adobe中文字体:AdobeSongStd、AdobeKaitiStd、AdobeHeitiStd、AdobeFangsongStd。
	\item TeX技能:尽管提供了对模板的必要说明,但这不是一份“ \LaTeX 入门文档”。在使用前请先通读其他入门文档。
	\item 针对Windows用户的额外需求:学位论文模本分别使用git和GNUMake进行版本控制和构建,建议从Cygwin安装这两个工具。
\end{itemize}

\subsection{准备工作}
\label{sec:requirements}

要使用这个模板撰写学位论文,需要在\emph{TeX系统}、\emph{中英文字体}、\emph{TeX技能}上有所准备。

\begin{itemize}[noitemsep,topsep=0pt,parsep=0pt,partopsep=0pt]
	\item 系统:所使用的系统要支持 引擎,且带有ctex 2.x宏包,以2015年的\emph{完整}TeXLive、MacTeX发行版为佳。
	\item 中英文字体:操作系统中需要安装TeX Gyre Termes字体和四款Adobe中文字体:AdobeSongStd、AdobeKaitiStd、AdobeHeitiStd、AdobeFangsongStd。
	\item TeX技能:尽管提供了对模板的必要说明,但这不是一份“ \LaTeX 入门文档”。在使用前请先通读其他入门文档。
	\item 针对Windows用户的额外需求:学位论文模本分别使用git和GNUMake进行版本控制和构建,建议从Cygwin安装这两个工具。
\end{itemize}

\subsection{准备工作}
\label{sec:requirements}

要使用这个模板撰写学位论文,需要在\emph{TeX系统}、\emph{中英文字体}、\emph{TeX技能}上有所准备。

\begin{itemize}[noitemsep,topsep=0pt,parsep=0pt,partopsep=0pt]
	\item 系统:所使用的系统要支持 引擎,且带有ctex 2.x宏包,以2015年的\emph{完整}TeXLive、MacTeX发行版为佳。
	\item 中英文字体:操作系统中需要安装TeX Gyre Termes字体和四款Adobe中文字体:AdobeSongStd、AdobeKaitiStd、AdobeHeitiStd、AdobeFangsongStd。
	\item TeX技能:尽管提供了对模板的必要说明,但这不是一份“ \LaTeX 入门文档”。在使用前请先通读其他入门文档。
	\item 针对Windows用户的额外需求:学位论文模本分别使用git和GNUMake进行版本控制和构建,建议从Cygwin安装这两个工具。
\end{itemize}

\subsection{准备工作}
\label{sec:requirements}

要使用这个模板撰写学位论文,需要在\emph{TeX系统}、\emph{中英文字体}、\emph{TeX技能}上有所准备。

\begin{itemize}[noitemsep,topsep=0pt,parsep=0pt,partopsep=0pt]
	\item 系统:所使用的系统要支持 引擎,且带有ctex 2.x宏包,以2015年的\emph{完整}TeXLive、MacTeX发行版为佳。
	\item 中英文字体:操作系统中需要安装TeX Gyre Termes字体和四款Adobe中文字体:AdobeSongStd、AdobeKaitiStd、AdobeHeitiStd、AdobeFangsongStd。
	\item TeX技能:尽管提供了对模板的必要说明,但这不是一份“ \LaTeX 入门文档”。在使用前请先通读其他入门文档。
	\item 针对Windows用户的额外需求:学位论文模本分别使用git和GNUMake进行版本控制和构建,建议从Cygwin安装这两个工具。
\end{itemize}

\subsection{准备工作}
\label{sec:requirements}

要使用这个模板撰写学位论文,需要在\emph{TeX系统}、\emph{中英文字体}、\emph{TeX技能}上有所准备。

\begin{itemize}[noitemsep,topsep=0pt,parsep=0pt,partopsep=0pt]
	\item 系统:所使用的系统要支持 引擎,且带有ctex 2.x宏包,以2015年的\emph{完整}TeXLive、MacTeX发行版为佳。
	\item 中英文字体:操作系统中需要安装TeX Gyre Termes字体和四款Adobe中文字体:AdobeSongStd、AdobeKaitiStd、AdobeHeitiStd、AdobeFangsongStd。
	\item TeX技能:尽管提供了对模板的必要说明,但这不是一份“ \LaTeX 入门文档”。在使用前请先通读其他入门文档。
	\item 针对Windows用户的额外需求:学位论文模本分别使用git和GNUMake进行版本控制和构建,建议从Cygwin安装这两个工具。
\end{itemize}

\subsection{准备工作}
\label{sec:requirements}

要使用这个模板撰写学位论文,需要在\emph{TeX系统}、\emph{中英文字体}、\emph{TeX技能}上有所准备。

\begin{itemize}[noitemsep,topsep=0pt,parsep=0pt,partopsep=0pt]
	\item 系统:所使用的系统要支持 引擎,且带有ctex 2.x宏包,以2015年的\emph{完整}TeXLive、MacTeX发行版为佳。
	\item 中英文字体:操作系统中需要安装TeX Gyre Termes字体和四款Adobe中文字体:AdobeSongStd、AdobeKaitiStd、AdobeHeitiStd、AdobeFangsongStd。
	\item TeX技能:尽管提供了对模板的必要说明,但这不是一份“ \LaTeX 入门文档”。在使用前请先通读其他入门文档。
	\item 针对Windows用户的额外需求:学位论文模本分别使用git和GNUMake进行版本控制和构建,建议从Cygwin安装这两个工具。
\end{itemize}

\subsection{准备工作}
\label{sec:requirements}

要使用这个模板撰写学位论文,需要在\emph{TeX系统}、\emph{中英文字体}、\emph{TeX技能}上有所准备。

\begin{itemize}[noitemsep,topsep=0pt,parsep=0pt,partopsep=0pt]
	\item 系统:所使用的系统要支持 引擎,且带有ctex 2.x宏包,以2015年的\emph{完整}TeXLive、MacTeX发行版为佳。
	\item 中英文字体:操作系统中需要安装TeX Gyre Termes字体和四款Adobe中文字体:AdobeSongStd、AdobeKaitiStd、AdobeHeitiStd、AdobeFangsongStd。
	\item TeX技能:尽管提供了对模板的必要说明,但这不是一份“ \LaTeX 入门文档”。在使用前请先通读其他入门文档。
	\item 针对Windows用户的额外需求:学位论文模本分别使用git和GNUMake进行版本控制和构建,建议从Cygwin安装这两个工具。
\end{itemize}